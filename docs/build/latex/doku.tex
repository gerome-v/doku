%% Generated by Sphinx.
\def\sphinxdocclass{report}
\documentclass[letterpaper,10pt,english]{sphinxmanual}
\ifdefined\pdfpxdimen
   \let\sphinxpxdimen\pdfpxdimen\else\newdimen\sphinxpxdimen
\fi \sphinxpxdimen=.75bp\relax
\ifdefined\pdfimageresolution
    \pdfimageresolution= \numexpr \dimexpr1in\relax/\sphinxpxdimen\relax
\fi
%% let collapsable pdf bookmarks panel have high depth per default
\PassOptionsToPackage{bookmarksdepth=5}{hyperref}

\PassOptionsToPackage{warn}{textcomp}
\usepackage[utf8]{inputenc}
\ifdefined\DeclareUnicodeCharacter
% support both utf8 and utf8x syntaxes
  \ifdefined\DeclareUnicodeCharacterAsOptional
    \def\sphinxDUC#1{\DeclareUnicodeCharacter{"#1}}
  \else
    \let\sphinxDUC\DeclareUnicodeCharacter
  \fi
  \sphinxDUC{00A0}{\nobreakspace}
  \sphinxDUC{2500}{\sphinxunichar{2500}}
  \sphinxDUC{2502}{\sphinxunichar{2502}}
  \sphinxDUC{2514}{\sphinxunichar{2514}}
  \sphinxDUC{251C}{\sphinxunichar{251C}}
  \sphinxDUC{2572}{\textbackslash}
\fi
\usepackage{cmap}
\usepackage[T1]{fontenc}
\usepackage{amsmath,amssymb,amstext}
\usepackage{babel}



\usepackage{tgtermes}
\usepackage{tgheros}
\renewcommand{\ttdefault}{txtt}



\usepackage[Bjarne]{fncychap}
\usepackage{sphinx}

\fvset{fontsize=auto}
\usepackage{geometry}


% Include hyperref last.
\usepackage{hyperref}
% Fix anchor placement for figures with captions.
\usepackage{hypcap}% it must be loaded after hyperref.
% Set up styles of URL: it should be placed after hyperref.
\urlstyle{same}

\addto\captionsenglish{\renewcommand{\contentsname}{Contents:}}

\usepackage{sphinxmessages}
\setcounter{tocdepth}{1}



\title{doku}
\date{Aug 30, 2021}
\release{0.1}
\author{GV}
\newcommand{\sphinxlogo}{\vbox{}}
\renewcommand{\releasename}{Release}
\makeindex
\begin{document}

\ifdefined\shorthandoff
  \ifnum\catcode`\=\string=\active\shorthandoff{=}\fi
  \ifnum\catcode`\"=\active\shorthandoff{"}\fi
\fi

\pagestyle{empty}
\sphinxmaketitle
\pagestyle{plain}
\sphinxtableofcontents
\pagestyle{normal}
\phantomsection\label{\detokenize{index::doc}}



\chapter{doku}
\label{\detokenize{modules:doku}}\label{\detokenize{modules::doc}}

\section{doku package}
\label{\detokenize{doku:doku-package}}\label{\detokenize{doku::doc}}

\subsection{Submodules}
\label{\detokenize{doku:submodules}}

\subsection{doku.data\_builder module}
\label{\detokenize{doku:module-doku.data_builder}}\label{\detokenize{doku:doku-data-builder-module}}\index{module@\spxentry{module}!doku.data\_builder@\spxentry{doku.data\_builder}}\index{doku.data\_builder@\spxentry{doku.data\_builder}!module@\spxentry{module}}\index{BaseDataBuilder (class in doku.data\_builder)@\spxentry{BaseDataBuilder}\spxextra{class in doku.data\_builder}}

\begin{fulllineitems}
\phantomsection\label{\detokenize{doku:doku.data_builder.BaseDataBuilder}}\pysigline{\sphinxbfcode{\sphinxupquote{class }}\sphinxcode{\sphinxupquote{doku.data\_builder.}}\sphinxbfcode{\sphinxupquote{BaseDataBuilder}}}
\sphinxAtStartPar
Bases: \sphinxcode{\sphinxupquote{object}}

\sphinxAtStartPar
Write one\sphinxhyphen{}line summary of this class.

\sphinxAtStartPar
Write multi\sphinxhyphen{}line summary of this class. Whatever you want to write you
can write it here.
\index{att\_a (doku.data\_builder.BaseDataBuilder attribute)@\spxentry{att\_a}\spxextra{doku.data\_builder.BaseDataBuilder attribute}}

\begin{fulllineitems}
\phantomsection\label{\detokenize{doku:doku.data_builder.BaseDataBuilder.att_a}}\pysigline{\sphinxbfcode{\sphinxupquote{att\_a}}}
\sphinxAtStartPar
description of this attribute.
\begin{quote}\begin{description}
\item[{Type}] \leavevmode
\sphinxAtStartPar
int

\end{description}\end{quote}

\end{fulllineitems}

\index{method\_a() (doku.data\_builder.BaseDataBuilder method)@\spxentry{method\_a()}\spxextra{doku.data\_builder.BaseDataBuilder method}}

\begin{fulllineitems}
\phantomsection\label{\detokenize{doku:doku.data_builder.BaseDataBuilder.method_a}}\pysiglinewithargsret{\sphinxbfcode{\sphinxupquote{method\_a}}}{\emph{\DUrole{n}{path}}}{}~\begin{quote}\begin{description}
\item[{Parameters}] \leavevmode
\sphinxAtStartPar
\sphinxstyleliteralstrong{\sphinxupquote{path}} (\sphinxstyleliteralemphasis{\sphinxupquote{str}}) \textendash{} path of the file to read.

\end{description}\end{quote}

\end{fulllineitems}


\end{fulllineitems}



\subsection{doku.model\_builder module}
\label{\detokenize{doku:module-doku.model_builder}}\label{\detokenize{doku:doku-model-builder-module}}\index{module@\spxentry{module}!doku.model\_builder@\spxentry{doku.model\_builder}}\index{doku.model\_builder@\spxentry{doku.model\_builder}!module@\spxentry{module}}\index{BaseModelBuilder (class in doku.model\_builder)@\spxentry{BaseModelBuilder}\spxextra{class in doku.model\_builder}}

\begin{fulllineitems}
\phantomsection\label{\detokenize{doku:doku.model_builder.BaseModelBuilder}}\pysigline{\sphinxbfcode{\sphinxupquote{class }}\sphinxcode{\sphinxupquote{doku.model\_builder.}}\sphinxbfcode{\sphinxupquote{BaseModelBuilder}}}
\sphinxAtStartPar
Bases: \sphinxcode{\sphinxupquote{object}}

\sphinxAtStartPar
Write one\sphinxhyphen{}line summary of this class.

\sphinxAtStartPar
Write multi\sphinxhyphen{}line summary of this class. Whatever you want to write you
can write it here.
\index{att\_a (doku.model\_builder.BaseModelBuilder attribute)@\spxentry{att\_a}\spxextra{doku.model\_builder.BaseModelBuilder attribute}}

\begin{fulllineitems}
\phantomsection\label{\detokenize{doku:doku.model_builder.BaseModelBuilder.att_a}}\pysigline{\sphinxbfcode{\sphinxupquote{att\_a}}}
\sphinxAtStartPar
description of this attribute.
\begin{quote}\begin{description}
\item[{Type}] \leavevmode
\sphinxAtStartPar
int

\end{description}\end{quote}

\end{fulllineitems}

\index{method\_a() (doku.model\_builder.BaseModelBuilder method)@\spxentry{method\_a()}\spxextra{doku.model\_builder.BaseModelBuilder method}}

\begin{fulllineitems}
\phantomsection\label{\detokenize{doku:doku.model_builder.BaseModelBuilder.method_a}}\pysiglinewithargsret{\sphinxbfcode{\sphinxupquote{method\_a}}}{\emph{\DUrole{n}{path}}}{}~\begin{quote}\begin{description}
\item[{Parameters}] \leavevmode
\sphinxAtStartPar
\sphinxstyleliteralstrong{\sphinxupquote{path}} (\sphinxstyleliteralemphasis{\sphinxupquote{str}}) \textendash{} path of the file to read.

\end{description}\end{quote}

\end{fulllineitems}


\end{fulllineitems}



\subsection{doku.utils module}
\label{\detokenize{doku:module-doku.utils}}\label{\detokenize{doku:doku-utils-module}}\index{module@\spxentry{module}!doku.utils@\spxentry{doku.utils}}\index{doku.utils@\spxentry{doku.utils}!module@\spxentry{module}}\index{func\_a() (in module doku.utils)@\spxentry{func\_a()}\spxextra{in module doku.utils}}

\begin{fulllineitems}
\phantomsection\label{\detokenize{doku:doku.utils.func_a}}\pysiglinewithargsret{\sphinxcode{\sphinxupquote{doku.utils.}}\sphinxbfcode{\sphinxupquote{func\_a}}}{\emph{\DUrole{n}{arg1}}}{}~\begin{quote}\begin{description}
\item[{Parameters}] \leavevmode
\sphinxAtStartPar
\sphinxstyleliteralstrong{\sphinxupquote{arg1}} (\sphinxstyleliteralemphasis{\sphinxupquote{int}}) \textendash{} description of arg1.

\item[{Returns}] \leavevmode
\sphinxAtStartPar
description of return val.

\item[{Return type}] \leavevmode
\sphinxAtStartPar
bool

\end{description}\end{quote}

\end{fulllineitems}

\index{func\_b() (in module doku.utils)@\spxentry{func\_b()}\spxextra{in module doku.utils}}

\begin{fulllineitems}
\phantomsection\label{\detokenize{doku:doku.utils.func_b}}\pysiglinewithargsret{\sphinxcode{\sphinxupquote{doku.utils.}}\sphinxbfcode{\sphinxupquote{func\_b}}}{\emph{\DUrole{n}{arg1}}}{}~\begin{quote}\begin{description}
\item[{Parameters}] \leavevmode
\sphinxAtStartPar
\sphinxstyleliteralstrong{\sphinxupquote{arg1}} (\sphinxstyleliteralemphasis{\sphinxupquote{int}}) \textendash{} description of arg1.

\item[{Returns}] \leavevmode
\sphinxAtStartPar
description of return val.

\item[{Return type}] \leavevmode
\sphinxAtStartPar
bool

\end{description}\end{quote}

\end{fulllineitems}



\subsection{Module contents}
\label{\detokenize{doku:module-doku}}\label{\detokenize{doku:module-contents}}\index{module@\spxentry{module}!doku@\spxentry{doku}}\index{doku@\spxentry{doku}!module@\spxentry{module}}

\chapter{Indices and tables}
\label{\detokenize{index:indices-and-tables}}\begin{itemize}
\item {} 
\sphinxAtStartPar
\DUrole{xref,std,std-ref}{genindex}

\item {} 
\sphinxAtStartPar
\DUrole{xref,std,std-ref}{modindex}

\item {} 
\sphinxAtStartPar
\DUrole{xref,std,std-ref}{search}

\end{itemize}


\renewcommand{\indexname}{Python Module Index}
\begin{sphinxtheindex}
\let\bigletter\sphinxstyleindexlettergroup
\bigletter{d}
\item\relax\sphinxstyleindexentry{doku}\sphinxstyleindexpageref{doku:\detokenize{module-doku}}
\item\relax\sphinxstyleindexentry{doku.data\_builder}\sphinxstyleindexpageref{doku:\detokenize{module-doku.data_builder}}
\item\relax\sphinxstyleindexentry{doku.model\_builder}\sphinxstyleindexpageref{doku:\detokenize{module-doku.model_builder}}
\item\relax\sphinxstyleindexentry{doku.utils}\sphinxstyleindexpageref{doku:\detokenize{module-doku.utils}}
\end{sphinxtheindex}

\renewcommand{\indexname}{Index}
\printindex
\end{document}